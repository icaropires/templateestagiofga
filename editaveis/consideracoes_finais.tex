\chapter[Considerações Finais]{Considerações Finais}

Durante o desenvolvimento do plugin foram trabalhados vários aspectos de conhecimento, principalmente na aplicação 
de conceitos que antes eu só conhecia na teoria. Também pude evoluir bastante a noção de como a complexidade de coisas
que antes pareciam ser simples se complicam dependendo do ambiente, tecnologias e outros aspectos 
que um projeto real exige.

Também foi uma experiência bastante interessante ter que ler a especificação de diversos protocolos, nacionais e internacionais. E, com
isso, entender melhor como eles interagem, como eles foram concebidos e a experiência de implementá-los diretamente da especificação.

Outro aspecto relativamente particular desse projeto, foi ter que lidar com a falta de várias fontes de informação, comum em outros
projetos. Neste, em muitos momentos, não havia fontes de informação além da especificação, que as vezes deixava dúvidas.

Foi um aprendizado bastante grande também da linguagem de programação utilizada: C++. E também de ter que realizar muitas implementações
num nível de abstração bem mais baixo do que o encontrado na maioria dos outros projetos que realizei na faculdade.

Além de aspectos de implementação, também entendi melhor como funciona a ICP-Brasil[5] e empresas relacionadas ao tema, parte disso devido
às parcerias para o fornecimento de exemplos de arquivos assinados. Essas experiências esclareceram bastante o cenário nacional sobre o tema para mim.
