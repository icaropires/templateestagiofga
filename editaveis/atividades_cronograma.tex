\chapter[Atividades e Cronograma]{Atividades e Cronograma}

\section{Atividades Desenvolvidas}

\subsection{Estudo inicial dos padrões da ICP-Brasil}

A ICP-Brasil dispõe dispõe de diversas regulamentações relativas a validade de Assinaturas digitais no Brasil. Para quem já
estudou o básico sobre segurança digital e assinaturas digitais, esses são os documentos iniciais de estudo para implementação
de assinaturas digitais no Brasil.

\subsection{Estudo inicial do SDK da Adobe}

O SDK da Adobe fornece uma estrutura inicial de projeto, além de uma biblioteca que deve ser utilizada para desenvolvimento de produtos
que interagem com os sesus. E, Visto que o produto final é um plugin que seja executado em produtos Adobe, é fundamental para execução do projeto
esse estudo do SDK.

\subsection{Realização de verificação manual de assinatura}

Nessa etapa o objetivo é que à partir de assinaturas já validas, eu conseguisse interagir com suas estruturas, utilizando qualquer ferramenta
que fosse mais conveniente. É a etapa em que se valida o que foi estudado na teoria e entende como a estrutura funciona à nível tecnólogico.

\subsection{Instalação e conhecimento do ambiente de desenvolvimento}

É uma etapa comum a todo projeto de desenvolvimento de software. Nela, visto que é essa é a segunda etapa de um projeto que já estava em execução,
já existia um ambiente de desenvolvimento montado para a compilação e execução de plugins da Adobe e utilizando tecnologias específicas do projeto.
Já existiam instruções parcialmente completas de como executar essa instalação do ambiente.

Além da instalação do ambiente, também foi importante uma compreensão melhor de como ele foi montado para que fosse possível resolver alguns problemas
e realizar algumas alterações.

\subsection{Implementação de assinatura básica}

Utilizando as tecnologias específicas do projeto e já gerando um plugin funcional, nessa etapa o objetivo é de construir uma assinatura básica,
não necessariamente que já atenda a tudo solicitado pela ICP-Brasil para uma assinatura ADRB.

\subsection{Impletação do padrão ADRB}

O padrão de assinaturas digitais ADRB foi instituído pela ICP-Brasil como o padrão mais básico válido e possui apenas referências básicas.
Ao término dessa etapa, a assinatura básica implementada na etapa anterior tem que atender a todos os requisitos da ICP-Brasil.

\subsection{Implementação do padrão ADRT}

Mais completo que o padrão ADRB, também possui referências de tempo. Ao final dessa etapa é esperado que o plugin seja capaz de realizar assinaturas
ADRT válidas.

\subsection{Implementação do padrão ADRC}

Mais completo que o padrão ADRT, o ADRC possui referências completas, de acordo com a ICP-Brasil. Ao final dessa etapa o plugin deve ser capaz de gerar
assinaturas ADRC válidas.

\subsection{Implementação do padrão ADRA}

Mais completo que o padrão ADRC, o ADRA possui referências para arquivamento. Ao final dessa etapa o plugin deve ser capaz de gerar assinaturas
ADRA válidas.

\subsection{Otimização de integração com o verificador}

O plugin já possuía um verificador das assinaturas. Nessa etapa deve-se integrar melhor a lógica que compõe ambos os módulos (verificador e assinador)
com o objetivo de melhorar a consistência entre verificação e validação, melhorar métricas de código e identificar possiveis enganos em ambos os módulos.

\subsection{Ajustes baseados em feedback do ITI}

Com o plugin quase finalizado, o ITI fez uma avaliação de pontos que ainda deveriam ser melhorados e baseados nesses pontos, correções são feitas nessa etapa.

\subsection{Otimizações finais}

Com verificações e assinaturas consistentes, nessa etapa deve-se realizar otimizações de memória e desempenho do plugin.

\section{Cronograma de Execução}
\pagebreak

\begin{table}[]
\begin{tabular}{|c|l|lll}
\cline{1-2}
\textbf{Período} & \multicolumn{1}{c|}{\textbf{Atividade}} &  &  &  \\ \cline{1-2}
09/06/2018 - 16/06/2018  & Estudo inicial dos padrões da ICP-Brasil &  &  &  \\ \cline{1-2}
17/06/2018 - 26/06/2018  & Estudo inicial do SDK da Adobe  &  &  &  \\ \cline{1-2}
27/06/2018 - 12/07/2018  & Realização de verificação manual de assinatura  &  &  &  \\ \cline{1-2}
13/07/2018 - 27/07/2018  & Instalação e conhecimento do ambiente de desenvolvimento &  &  &  \\ \cline{1-2}
28/07/2018 - 31/08/2018  & Implementação de assinatura básica &  &  &  \\ \cline{1-2}
01/09/2018 - 30/11/2018  & Implementação do padrão ADRB &  &  &  \\ \cline{1-2}
01/12/2018 - 31/01/2019  & Implementação do padrão ADRT &  &  &  \\ \cline{1-2}
01/02/2019 - 31/03/2019  & Implementação do padrão ADRC &  &  &  \\ \cline{1-2}
01/04/2019 - 31/04/2019  & Implementação do padrão ADRA &  &  &  \\ \cline{1-2}
01/05/2019 - 31/05/2019  & Otimização de integração com o verificador &  &  &  \\ \cline{1-2}
01/06/2019 - 19/06/2019  & Ajustes baseados em feedback do ITI &  &  &  \\ \cline{1-2}
20/06/2019 - 08/07/2019  & Otimizações finais &  &  &  \\ \cline{1-2}
\end{tabular}
\end{table}
